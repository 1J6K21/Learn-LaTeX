\documentclass{article}
\usepackage{amssymb}
\usepackage{amsmath}
\usepackage[style=authoryear,backend=biber]{biblatex}
\addbibresource{week2.bib} 

\author{Jonathan Kalsky}
\date{Jan 22 2026}
\title{Week 2 Predicate Logic}
\linespread{1.5}

\begin{document}
\maketitle
\vspace{1cm}
\section*{Negate “If a user is active, at least one network link will be available.”}

p : "A user is Active" \\
q : "A network link will be available"\\

\noindent Starting with: 
$$
p \to q
$$
and $p \to q \equiv \lnot p \lor q(x)$\\

\noindent negation: \\
\begin{center}
    
    $\lnot (p \to q)$ \\
    \vspace{0.5cm}
    $\lnot (p \to q) \equiv p \land \lnot q(x)$\\
    \vspace{0.5cm}
    $\therefore$ if a user is active, and all newtwork links are not available.
\end{center}

\section*{Why $\forall  x \exists  y P(x, y) \neq \exists x \forall y P(x, y)$ ?}

\begin{center}
    
    $\forall x \exists y (x + y = 0)$ is true\\
    All x's have an additive inverse\\
    \vspace{0.5cm}
    $\exists x \forall y (x + y = 0)$ is false!\\
    All y's are not the additive inverses of a single x
\end{center}

\vspace{1cm}

\section*{The norm/convention}
\begin{itemize}
    \item Negations only occur and are written immediately before predicates, and not before quantifiers.
    Quanitifers: $\forall \: or \: \exists$.
    \item They have the highest priority over anything else (they affect the proposition that comes right after).
\end{itemize}


\vspace{1cm}
\noindent Negate at the boundaries first:

\begin{align*}
    \lnot (\forall x \exists y (x \cdot y = 1))
        &\equiv \exists x \lnot(\exists y (x \cdot y = 1)) \\
        &\equiv \exists x \forall y \lnot(x \cdot y = 1) \\
        &\equiv \exists x \forall y (x \cdot y \neq 1)
\end{align*}

\section*{Bound/Free Variables}
``A bound variable is restricted by a quantifier (\(\forall \) or \(\exists \)) within an expression, making its value determined by the context (e.g., \(\forall xP(x)\)). A free variable is not constrained by a quantifier, allowing it to take any value within its domain. Bound variables act as placeholders, while free variables often depend on external definitions. "
\cite{fiveable2025freebound}

\end{document}