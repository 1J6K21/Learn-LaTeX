\documentclass{article}
\usepackage{amssymb}
\usepackage{amsfonts}
\usepackage{amsmath}
\usepackage[style=authoryear,backend=biber]{biblatex}
\addbibresource{week2.bib} 

\author{Jonathan Kalsky}
\date{Feb. 10 2026}
\title{Week 5 Sets}
\linespread{1.5}

\begin{document}
\maketitle
\vspace{1cm}

\section*{Sets}

%left align titles pls
\begin{itemize}
    \item[\textbf{Set}]
    \item Set $S = \{...\} $, empty set $\emptyset$, singleton $\{a\}$ 
    \item Super set, means parent set
    \item[\textbf{Subset}]
    \item A proper subset A, $(\subset)$ does not equal its superset B
    \item cannot equal: $\subset$ can equal: $\subseteq$
    \item[\textbf{Power Set}]
    \item The power set of a set S is the set of all subsets of S. Notation: $P(S)$
    \item Power set of empty set ($P(\emptyset)$) is $\{\emptyset\} or \{\{\}\}$
    \item[\textbf{cardinality}]
    \item all powersets will have a minimum cardinality of 1 bc it could be empty (not for all subsets which can have $\{\}$)

\end{itemize}

Idempotent = same in latin


$|S| = cardinality of s, |P(s)| = 2^n$

like binary of whats included where 000 => empty set

Union U is the elements in both sets $\cup$  A or B
interesection $\cap$ is A and B

B - A = B and not A
$A\complement = U - A$ universe without A. also written $\overline{A}$

Cartesian product of sets A and B is the set of all ordered pairs (a, b) where a is in set A and b is in set B

$A \times B = \{(a,b) | (a \in A) \land (b \in B)\}$

the order in which the sets are multiplied matters. because if the sets are (1) and (2), then (1,2) is not (2,1) in other words, (a,b) is unique of (b,a)
result amount of elements is the product of the amount of elements in both sets. Like iterate through A for each B = AxB
A relation R from A to B is a subset of AxB. It is like a filter subset via lambda

A is the domain of R and B is the co-domain of R

Fucntions vs. Relations ... TBC
\end{document}