\documentclass{article}
\usepackage{amssymb}
\usepackage{amsfonts}
\usepackage{amsmath}

\author{Jonathan Kalsky}
\date{Feb. 13 2026}
\title{\normalsize CSCE 222\\ \Large Homework 2}
\linespread{1.5}

\begin{document}
\maketitle
\tableofcontents
\vspace{1cm}

%Section 1.6
\section{Section 1.6}
\section*{Question 14}
\subsection*{a)}
p = "In the class"\\
q = "Owns a red convertible"\\
r = "Got a speeding ticket"\\
$\forall x(q(x) \to r(x))$ premise\\
$q(Linda) \to r(Linda)$ Linda follows this premise\\
$p(Linda) \land q(Linda)$ Premise\\
$q(Linda)$ Simplification\\
$r(Linda)$ Modus ponens\\
Due to Linda, there is at least one person who has gotten a speeding ticket in this class

\subsection*{b)}
p = "Taken a course in discrete mathematics"\\
q = "Can take a course in algorithms"\\
$\forall x (p(x) \to q(x))$\\
$p(M) \rightarrow q(M)$, $p(R) \rightarrow q(R)$, $p(V) \rightarrow q(V)$, $p(K) \rightarrow q(K)$ Universaly True \\
$p(M) \land p(A) \land p(R) \land p(V) \land p(K)$ Premise\\
$q(M) \land q(A) \land q(R) \land q(V) \land q(K)$ Modus Ponens\\
Because all 5 took discrete math, this implies all 5 are eligible for algorithms

\subsection*{c)}
p = "produced by John Sayles"\\
q = "Is wonderful"\\
$\forall x (p(x) \to q(x))$\\
$p(Coal)$ Premise\\
$p(coal) \to q(Coal)$ Universal Application\\
$q(Coal)$ Modus Ponens\\
Because the movie about coal miners is produced by John Sayles, the movie is wonderful; thus, there is a wonderful movie about coal miners.

\subsection*{d)}

p = "Been to France"\\
q = "Visited the Louvre"\\
$\forall x (p(x) \to q(x))$\\
$p(S)$ Premise\\
$p(S) \to q(S)$ Universal Application\\
$q(S)$ Modus Ponens\\
Given there is someone who has been to France in the class, someone visited the Louvre.

\section*{Question 24}
The error is on step 3, because you cannot simplify $P(c) \lor Q(c)$ to $P(c)$

\section*{FRQ}
p = "There is an undeclared variable"\\
q = "There is a syntax error in the first five lines"\\
r = "There is a missing semicolon"\\
s = "Variable name is misspelled"\\
$q \to (r \lor s)$ Premise\\
$p \lor q$ Premise\\
$\lnot r $ Premise \\
$\lnot s $ Premise \\
$\lnot (r \lor s) \to \lnot q \equiv (\lnot r \land \lnot s) \to \lnot q$ Modus Tollens\\
$\lnot q$\\
$p$ Disjunctive Syllogism\\
Given there is no missing semicolon or misspelled variable name, there is not a syntax error in the first 5 lines. This then also concludes that there is an undeclared variable.

\section{Section 1.7}

\section*{Question 18}

$\forall m,n,k \in \mathbb{Z}( mn = 2k \iff$mn is even) By definition.\\
Additionally, if m is even or n is even $\forall a,b \in \mathbb{Z},$\\
$(m = 2a \land n = 2b + 1)\lor(m = 2b + 1 \land n = 2a) \to mn = 2(a(2b+1))$\\ Through association, mn is even by definition.\\
$mn \in \mathbb{Z}$ because $k \in \mathbb{Z}$, multiplication is closed under $\mathbb{Z}$\\
Therefore, if m or n are even integers, mn is even.

\section*{Question 20}
p = "n is an integer"
q = "is even"
\section*{a) By contraposition}
$p(n) \land q(3n+2) \to q(n)$ To Prove\\
$\lnot q(n) \to \lnot p(n) \lor \lnot q(3n+2)$ Contraposition\\
Addition and multiplication is closed under the set of integers so $p(n)$\\
Now prove: $\lnot q(n) \to \lnot q(3n+2)$ Logical Reduction\\
If q(n), n can be written in the form of 2k+1 such that k is an integer\\
$\lnot q(2k+1) \to \lnot q(3(2k+1)+2 = 2(3k+2) + 1)$\\
This relationship is tautological\\
Through contraposition, we can prove the statement is true.



\section*{b) By contradiction}
$p(n) \land q(3n+2) \to q(n)$ To Prove\\
Assume $\lnot q(n)$ By Contradiction\\
n can be written as 2k + 1  such that k is an integer\\
3n+2 = 3(2k+1) + 2 = 6k + 5 = 2(3k + 2) + 1\\
This implies $\lnot q(3n+2)$ which contradicts the premise and assumption of q(3n+2)


\section*{Question 34}

x is rational $\iff x = \frac{p}{q} \mid p,q \in \mathbb{Z}, q \neq 0, gcd(p,q)=1 $ \\
x/2 = $\frac{2p_x}{q_x}$, keeping the gcd the same, p and q as integers, and q $\neq 0$ \\
3x-1 = $\frac{3p_x}{q_x} - 1$\\
This remains rational because a rational - 1 equals the whole + fractions inverse(which is rational).\\
If x is rational, all 3 are valid rational expressions\\
x is rational $\equiv$ x/2 is rational $\equiv$ 3x-1 is rational\\

\section*{FRQ}

Prove: $x \in \mathbb{Z^+} \to (x = 2k \iff 7x+4 = 2k | k \in \mathbb{Z^+})$\\
7x+4 = 2(3x+2) + x\\
This implies 7x+4 is only a multiple of 2 if x is also a multiple of 2, where x is a positive integer.

\section*{FRQ}
R = "The square root of any irrational number is irrational."
\section*{a)}
$\lnot R =$ "The square root of any irrational number is rational."
\section*{b)}
prove: $\sqrt{irrational} = irrational$\\
Assume opposite: $\sqrt{irrational} = rational$\\
$\sqrt{irrational} = \frac{p}{q}$ Definition of a rational number\\
$irrational = (\frac{p}{q})^2$\\
This expresses an irrational number as a rational by definition which is a contradiction. Therefore, the square root of an irrational number is irrational.

\section{Section 2.1} 

\section*{Question 46}
If A = $(a_1, a_2)$ and B = $(b_1,b_2)$\\
AxB is a set of $a_1$ and $a_2$ with all combinations of B, in that order.\\
Switching the order of AxB implies that the element pairs of the product set will not be the same\\
Multiplication of sets is not commutative\\
This fails to equate AxB and BxA under equivalence of sets by definition.\\
Therefore the order of multiplication matters, and the resulting set is not necessarily equivalent if switched. (AxB) ×(C x D) and A x (B x C) ×D are not the same.

\section*{Question 48}
Prove $A \times B = A \times C \to B = C$\\
A, B, and C are nonempty sets $\mid$ Premise\\
$A\times B = \{(a,b)\} | a\in A \land b \in B \mid$ Cartesian Product\\
$A\times C = \{(a,c)\} | a\in A \land c \in C \mid$ Cartesian Product\\
$\forall a,b,c ( (a,b) = (a,c) \to b = c) \to B = C \mid$ Definition of Set Equality\\
All elements of B are in C, and all elements of C are in B, shown by the equality when taking the cartesian product with A.

\section{Section 2.2}
\section*{Question 20}
\section*{a)}
p = "element in A"
q = "element in B"
r = "element in C"\\
$(A\cup B \cup C) = (A\cup B) \cup C \mid$ Association Law\\
$(p \lor q) \lor r \mid$ Written as propositions\\
$(p \lor q) \to (p \lor q) \lor r \mid$ Tautological\\
Therefore $(A\cup B) \subseteq (A\cup B \cup C)$ because all elements of $(A\cup B \cup C)$ are in $(A\cup B) \cup C$

\section*{b)}
% (A ∩B ∩C) ⊆ (A ∩B)
$A \cap B \cap C \equiv (A \cap B) \cap C \mid$ Association Law\\
$x \in (A \cap B \cap C)) \iff x \in A \land x \in B \land x \in C \mid$ Meaning of Intersection\\
$x \in A \land x \in B \land x \in C \to x \in A \land x \in B \mid$ Simplification\\
$\therefore A \cap B \cap C \subseteq A \cap B$
\section*{c)}
% (A−B)−C ⊆ A−C.
$(A - B) - C \equiv (A - C) - B\mid$ Associative and Commutative Law\\
$(A - C) - B\equiv \{\forall x(x\in (A - C) \land (A - C) - B)\}\mid$ Equivalence of a set intersection\\
$\forall x(x\in (A - C) \land (A - C) - B) \to \forall x(x \in (A-C))$\\
All elements of this set have to be in A-C, thus it is a subset of A-C, therefore\\
$(A - B) - C  \subseteq A-C$

\section*{d)}
% (A−C) ∩(C−B) = ∅.
$\forall x(x \in (A-C) \to x \notin C) \mid$ Definition of set subtraction\\
$\forall x(x \in (C-B) \to x \in C) \mid$ Definition of set subtraction\\
$\forall x\lnot ((x\in (A-C)) \land (x \in (C-B))) \to (A-C) \cap (C-B) = \emptyset \mid$ Since by definition of interesection, for all elements there are no elements existing in both sets, the resulting set is empty by definition of an empty set.


\section*{e)}
%  (B−A) ∪(C−A)= (B ∪C)−A.
$(B-A) \cup (C-A) \equiv \{x \in B | x \notin A\} \cup \{x\in C | x \notin A\}\mid$Definition of Set Difference\\
$\equiv \{x \in B \cup C | x \notin A\}\mid$ Definition of Set Union\\
$\equiv (B \cup C)\cap A^c \mid$ Definition of a Complement\\
$\equiv (B \cup C) - A \mid$ Definition of Set Difference\\
$\therefore (B-A) \cup (C-A) \equiv (B \cup C)-A$


\end{document}