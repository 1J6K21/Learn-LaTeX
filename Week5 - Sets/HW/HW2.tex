\documentclass{article}
\usepackage{amssymb}
\usepackage{amsfonts}
\usepackage{amsmath}

\author{Jonathan Kalsky}
\date{Feb. 13 2026}
\title{Homework 2}
\linespread{1.5}

\begin{document}
\maketitle
\tableofcontents
\vspace{1cm}

%Section 1.6
\section{Section 1.6}
\section*{Question 14}
\subsection*{a)}
p = "In the class"\\
q = "Owns a red convertible"\\
r = "Got a speeding ticket"\\
$\forall x(q(x) \to r(x))$ premise\\
$p(x) \land q(x)$ Premise\\
$q(x)$ Simplification\\
$r(x)$ Modus ponens\\
Due to linda, there is at least one person who has gotten a speeding ticket in this class

\subsection*{b)}
p = "Taken a course in discrete mathematics"\\
q = "Can take a course in algorithms"\\
$\forall x (p(x) \to q(x))$\\
$p(M) \land p(A) \land p(R) \land p(V) \land p(K)$ Premise\\
$q(M) \land q(A) \land q(R) \land q(V) \land q(K)$ Modus Ponens\\
Because all 5 took discrete math, this implies all 5 are eligible for algorithms

\subsection*{c)}
p = "produced by John Sayles"\\
q = "Is wonderful"\\
$\forall x (p(x) \to q(x))$\\
$p(Coal)$ Premise\\
$q(Coal)$ Modus Ponens\\
Because the movie about coal miners is produced by John Sayles, the movie is wonderful; thus, there is a wonderful movie about coal miners.

\subsection*{d)}

p = "Been to France"\\
q = "Visited the Lovre"\\
$\forall x (p(x) \to q(x))$\\
$p(x)$ Premise\\
$q(x)$ Modus Ponens\\

\section*{Question 24}
The error is on step 3, because you cannot simplify $P(c) \lor Q(c)$ to $P(c)$

\section*{FRQ}
p = "There is an undeclared variable"\\
q = "There is a syntax error in the first five lines"\\
r = "There is a missing semicolon"\\
s = "Variable name is misspelled"\\
$q \to (r \lor s)$\\
$p \lor q$\\
$\lnot r $ Premise \\
$\lnot s $ Premise \\
$\lnot (r \lor s) \to \lnot q = (\lnot r \land \lnot s) \to \lnot q$ Modus Tollens\\
$\lnot q$\\
Given there is no missing semicolon or misspelled variable name, there is not a syntax error in the first 5 lines.

\section{Section 1.7}

\section*{Question 18}

$\forall m,n,k \in \mathbb{Z}, mn = 2k \iff$mn is even. By definition.\\
$(m = 2k \land n = 2k + 1)\lor(m = 2k + 1 \land n = 2k) \to mn = (2(k(2k+1))) \lor (2(k(2k)))$ Through association, mn is even by definition.\\
$mn \in \mathbb{Z}$ under multiplication since $k \in \mathbb{Z}$.\\
Therefore, if m or n are even integers, mn is even.

\section*{Question 20}
p = "n is an integer"
q = "is even"
\section*{a) By contraposition}
$p(n) \land q(3n+2) \to q(n)$ Premise\\
$\lnot q(n) \to \lnot p(n) \lor \lnot q(3n+2)$ Contraposition\\
(3n + 2) = (3n + 2(1)) Identity\\
2(1) is even by definition. Even + Even = Even, Odd + Even = Odd\\
Since: 2k + 2k = 2(2k) and (2k + 1) + (2k) = 2(2k) + 1\\
By the same reduction,
3n + 2 = 2(n+1) + n = n\\
$\lnot q(n) \equiv \lnot q(3n+2)$ By definition of even\\
This makes the condition true, and since the contrapositive is true, the original is true.



\section*{b) By contradiction}
$p(n) \land q(3n+2) \to q(n)$ Premise\\
Assume $\lnot q(n)$ By Contradiction\\
3n+2 = 3(2k+1) + 2 = 6k + 5 = 2(3k + 2) + 1\\
This implies $\lnot q(3n+2)$ which contradicts the premise


\section*{Question 34}

if rational x was equivalent to rational x/2, it could be written as $x = \frac{p}{q}$ and $x/2 = \frac{p}{q/2}$, where q is just another integer.\\
Then, 3x-1 can also be written as $\frac{3p}{q} -1 = \frac{3p-q}{q} $\\
x can take on any real number, so with different combinations of p and q,\\
x is rational $\equiv$ x/2 is rational $\equiv$ 3x-1 is rational\\

\section*{FRQ}

Prove: $x \in \mathbb{Z^+} \to (x = 2k \iff 7x+4 = 2k | k \in \mathbb{Z^+})$\\
7x+4 = 2(3x+2) + x\\
This implies 7x+4 is only a multiple of 2 if x is also a multiple of 2, where x is a positive integer.

\section*{FRQ}
R = "The square root of any irrational number is irrational."
\section*{a)}
$\lnot R =$ "The square root of any irrational number is rational."
\section*{b)}
prove: $\sqrt{irrational} = irrational$\\
Assume opposite: $\sqrt{irrational} = rational$\\
$\sqrt{irrational} = \frac{p}{q}$ Definition of a rational number\\
$irrational = (\frac{p}{q})^2$\\
$p,q \in \mathbb{Z}$, so $(\frac{p}{q})^2$ is likely to be rational, which is a contradiction.\\
By contradiction R is true, "The square root of any irrational number is irrational."

\section{Section 2.1} 

\section*{Question 46}
Use proofs?\\
Multiplication of sets is not commutative:
If A = $(a_1, a_2)$ and B = $(b_1,b_2)$\\
AxB is a set of $a_1$ and $a_2$ with all combinations of B, in that order.
Therefore the order of multiplication matters, and the resulting set is not necessarily equivalent if switched. (AxB) ×(C x D) and A x (B x C) ×D are not the same.

\section*{Question 48}
Prove or disprove that if A, B, and C are nonempty sets and A × B = A × C, then B = C.\\
A, B, and C are nonempty sets | Given\\
Prove $A \times B = A \times C \to B = C$\\
$A\times B = \{(a,b)\} | a\in A \land b \in B$\\
$A\times C = \{(a,c)\} | a\in A \land c \in C$\\
$\forall a,b,c ( (a,b) = (a,c) \to b = c) \to B = C$\\




\section{Section 2.2}
\section*{Question 20}

\end{document}