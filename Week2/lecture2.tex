\documentclass{article}
\usepackage{amssymb}
\usepackage{amsmath}
\author{Jonathan Kalsky}
\date{Jan 22 2026}
\title{Lecture 2 Examples}
\linespread{1.5}

\begin{document}
\maketitle
\vspace{1cm}
\section*{Negate “If a user is active, at least one network link will be available.”}

p : "A user is Active" \\
q : "A network link will be available"\\

\noindent Starting with: 
$$
p \to q
$$
and $p \to q \equiv \lnot p \lor q(x)$\\

\noindent negation: \\
\begin{center}
    
    $\lnot (p \to q)$ \\
    \vspace{0.5cm}
    $\lnot (p \to q) \equiv p \land \lnot q(x)$\\
    \vspace{0.5cm}
    $\therefore$ if a user is active, and all newtwork links are not available.
\end{center}

\section*{Why $\forall  x \exists  y P(x, y) \neq \exists x \forall y P(x, y)$ ?}

\begin{center}
    
    $\forall x \exists y (x + y = 0)$ is true\\
    All x's have an additive inverse\\
    \vspace{0.5cm}
    $\exists x \forall y (x + y = 0)$ is false!\\
    All y's are not the additive inverses of a single x
\end{center}

\vspace{1cm}

\section*{The norm/convention}
Negations only occur and are written immediately before predicates, and not before quantifiers.
Quanitifers: $\forall \: or \: \exists$

\vspace{1cm}
\noindent Negate at the boundaries first:

\begin{align*}
    \lnot (\forall x \exists y (x \cdot y = 1))
        &\equiv \exists x \lnot(\exists y (x \cdot y = 1)) \\
        &\equiv \exists x \forall y \lnot(x \cdot y = 1) \\
        &\equiv \exists x \forall y (x \cdot y \neq 1)
\end{align*}

\end{document}