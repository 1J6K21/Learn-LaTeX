\documentclass[10pt,landscape]{extarticle}
\usepackage[utf8]{inputenc}
\usepackage[margin=0.4in]{geometry}
\usepackage{multicol}
\usepackage{enumitem}
\usepackage{titlesec}
\usepackage{xcolor}
\usepackage{nopageno}
\usepackage{amsmath}
%Use the other repo to generate material context 
% Compact list settings
\setlist{nosep, leftmargin=*}
\setlist[itemize]{label=\textbullet}

% Compact section titles
\titleformat{\section}
  {\normalfont\large\bfseries\uppercase}{\thesection}{1em}{}
\titlespacing*{\section}{0pt}{2ex plus 1ex minus .2ex}{0.8ex plus .2ex}

\titleformat{\subsection}
  {\normalfont\bfseries}{\thesubsection}{1em}{}
\titlespacing*{\subsection}{0pt}{1ex plus 1ex minus .2ex}{0.5ex plus .2ex}

\setlength{\parindent}{0pt}
\setlength{\parskip}{2pt}
\setlength{\columnsep}{20pt}

\begin{document}

\begin{multicols*}{3}

\begin{center}
    \textbf{\Large GEOLOGY EXAM 1: COMPREHENSIVE GUIDE}
\end{center}

\section{Introduction to Geology}
\subsection{Basics of Geology}
\begin{itemize}
    \item \textbf{Geology}: The science of Earth's processes, composition, structure, and history. Interdisciplinary (Chem, Phys, Bio).
    \item \textbf{Physical Geology}: Study of Earth materials and internal/surface processes.
    \item \textbf{Historical Geology}: Study of Earth's origin and development through time.
    \item \textbf{Why Study?}: Resource dependence (minerals, energy, water), Hazards, Environmental health.
\end{itemize}

\subsection{Scientific Principles}
\begin{itemize}
    \item \textbf{Uniformitarianism}: "The present is the key to the past." Processes operating today (erosion, volcanism, gravity) operated in the past.
    \item \textbf{Scientific Method}: 
        \begin{enumerate}
            \item \textbf{Observation}: Data collection.
            \item \textbf{Hypothesis}: Tentative, testable explanation.
            \item \textbf{Theory}: Well-tested, widely accepted view (e.g., Plate Tectonics).
        \end{enumerate}
    \item \textbf{System}: Group of interacting parts (e.g., Earth System).
        \begin{itemize}
            \item \textbf{Open System}: Energy/matter flow in/out.
            \item \textbf{Feedback Mechanisms}: +/- loops.
        \end{itemize}
\end{itemize}

\subsection{Geology \& Society (Facts)}
\begin{itemize}
    \item \textbf{Resource Usage}: Each American uses ~3.5 million lbs of minerals/metals/fuels in a lifetime.
    \item \textbf{Population Bomb}: Exponential growth ($N = N_0 e^{kt}$ - you don't need to memorize the formula but know it's exponential).
    \item \textbf{Carrying Capacity}: Max population Earth can support without degradation.
    \item \textbf{Env. Disasters}:
        \begin{itemize}
            \item \textbf{BP Deepwater Horizon (2010)}: Largest marine oil spill. 152 days.
            \item \textbf{Tohoku Earthquake (2011)}: Mag 9.1. Waves up to 133 ft, 435 mph. Moved Japan 8 ft. Shifted Earth axis 10 inches. Increased rotation speed (1.8 microseconds/day).
        \end{itemize}
\end{itemize}

\section{Origins \& Time}
\subsection{Origin of the Universe \& Solar System}
\begin{itemize}
    \item \textbf{Big Bang}: ~13.7 Billion years ago (estimated 10-15 Ga). All matter/energy expanded from a point.
    \item \textbf{Nebular Theory}: Solar system evolved from a rotating cloud of gas/dust (Solar Nebula).
        \begin{itemize}
            \item Collapse $\rightarrow$ Spinning Disk $\rightarrow$ Proto-Sun $\rightarrow$ Planetesimals.
        \end{itemize}
    \item \textbf{Age of Earth/Solar System}: 4.6 Billion Years (4.6 Ga).
\end{itemize}

\subsection{Early Earth Evolution}
\begin{itemize}
    \item \textbf{Formation}: Accretion of high-velocity debris.
    \item \textbf{Molten Earth}: Heat from impacts and radioactive decay caused melting.
    \item \textbf{Differentiation}: Separation by density.
        \begin{itemize}
            \item \textbf{Iron/Nickel} sank to center (Core).
            \item \textbf{Lighter rock} floated (Crust/Mantle).
            \item \textbf{Gases} escaped to form primitive atmosphere (Outgassing).
        \end{itemize}
\end{itemize}

\subsection{Geologic Time}
\begin{itemize}
    \item \textbf{Scale}: Eons $\rightarrow$ Eras $\rightarrow$ Periods $\rightarrow$ Epochs.
    \item \textbf{Precambrian}: Hadean, Archean, Proterozoic (88\% of Earth history).
    \item \textbf{Phanerozoic}: "Visible Life" (Paleozoic, Mesozoic, Cenozoic).
    \item \textbf{Analogy}: If Earth history = 1 year, Humans appear at 11:49 PM on Dec 31st.
\end{itemize}

\section{Earth's Structure}
\subsection{Compositional Layers (Chemistry)}
\begin{itemize}
    \item \textbf{Crust}: Thin, rocky outer skin.
        \begin{itemize}
            \item \textbf{Oceanic}: Thin (~7 km), Denser (3.0 g/cm$^3$), Basaltic (Mafic). Young (<180 Ma).
            \item \textbf{Continental}: Thick (35-70 km), Less Dense (2.7 g/cm$^3$), Granitic (Felsic). Old.
        \end{itemize}
    \item \textbf{Mantle}: Solid, rocky shell. 82\% of Earth's volume. \textbf{Peridotite} (Ultramafic).
    \item \textbf{Core}: Iron-Nickel alloy. Very dense.
\end{itemize}

\subsection{Physical Layers (Properties)}
\begin{itemize}
    \item \textbf{Lithosphere}: Crust + Uppermost Mantle. Cool, rigid, brittle. Forms Tectonic Plates.
    \item \textbf{Asthenosphere}: Weak, ductile (plastic) layer in upper mantle. Allows plates to move.
    \item \textbf{Mesosphere}: Lower mantle. Solid, strong.
    \item \textbf{Outer Core}: \textbf{Liquid} Fe-Ni. Generates Earth's \textbf{Magnetic Field}.
    \item \textbf{Inner Core}: \textbf{Solid} Fe-Ni. Immense pressure prevents melting despite high temps.
\end{itemize}

\subsection{Seismic Evidence}
\begin{itemize}
    \item \textbf{P-Waves}: Primary, Push-pull (compressional). Travel through solids, liquids, gases. Fastest.
    \item \textbf{S-Waves}: Secondary, Shear. Travel ONLY through solids. Stopped by Outer Core (proving it's liquid).
\end{itemize}

\section{Plate Tectonics}
\subsection{Development of the Theory}
\begin{itemize}
    \item \textbf{Continental Drift (1915)}: Alfred Wegener. Pangaea. Rejected due to lack of mechanism.
    \item \textbf{Plate Tectonics (1960s)}: Unified theory. Lithosphere broken into plates moving on asthenosphere.
    \item \textbf{Driving Forces}:
        \begin{itemize}
            \item \textbf{Slab Pull}: Main driver. Cold, dense slab sinks.
            \item \textbf{Ridge Push}: Gravity slides plate off high ridge.
            \item \textbf{Mantle Convection}: Heat transfer.
        \end{itemize}
    \item \textbf{Plate Rate}: Average ~3 cm/year (fingerprint growth). North American plate is relatively slow.
\end{itemize}

\subsection{Plate Boundaries}
\begin{itemize}
    \item \textbf{Divergent} ($\leftarrow \rightarrow$): Plates move apart. New crust created.
        \begin{itemize}
            \item \textbf{Features}: Mid-Ocean Ridges (seafloor spreading), Rift Valleys.
            \item \textbf{Melting}: Decompression Melting.
            \item \textbf{Examples}: Mid-Atlantic Ridge, East African Rift, Iceland.
        \end{itemize}
    \item \textbf{Convergent} ($\rightarrow \leftarrow$): Plates move together. Crust recycled/destroyed.
        \begin{itemize}
            \item \textbf{Ocean-Continent}: Subduction. Trench + Continental Volcanic Arc (e.g., Andes, Cascades).
            \item \textbf{Ocean-Ocean}: Subduction. Trench + Volcanic Island Arc (e.g., Japan, Aleutians).
            \item \textbf{Continent-Continent}: Collision. No subduction. Huge Mountains (e.g., Himalayas).
            \item \textbf{Melting}: Flux Melting (Water from slab lowers melting point).
        \end{itemize}
    \item \textbf{Transform} ($\uparrow \downarrow$): Plates slide past. No production/destruction.
        \begin{itemize}
            \item \textbf{Features}: Faults, Earthquakes. No volcanoes.
            \item \textbf{Example}: San Andreas Fault.
        \end{itemize}
\end{itemize}

\subsection{Evidence for Plate Tectonics}
\begin{itemize}
    \item \textbf{Paleomagnetism}: Iron minerals (Magnetite) align with magnetic field at \textbf{Curie Point} (~585$^\circ$C / 1800$^\circ$F).
    \item \textbf{Magnetic Reversals}: Symmetric "bar code" stripes of normal/reversed polarity on seafloor. Proves seafloor spreading.
    \item \textbf{Seafloor Age}: Youngest at ridges, oldest at trenches. Atlantic ocean doesn't exist >200 Ma.
    \item \textbf{Hot Spots}: Stationary mantle plumes (intraplate). Plate moves over them.
        \begin{itemize}
            \item Example: Hawaii. Chain shows direction of plate motion.
        \end{itemize}
    \item \textbf{GPS}: Direct measurement of motion.
\end{itemize}

\section{Minerals}
\subsection{Definition}
\begin{itemize}
    \item \textbf{Criteria}: 1) Naturally occurring, 2) Inorganic, 3) Solid, 4) Ordered crystalline structure, 5) Definite chemical composition.
    \item \textbf{Rock}: Solid aggregate of minerals.
    \item \textbf{Atom Basics}: Protons (+), Neutrons (0), Electrons (-). Bonding occurs to satisfy shells.
\end{itemize}

\subsection{Chemical Bonding}
\begin{itemize}
    \item \textbf{Ionic}: Transfer of electrons. Electrostatic attraction (e.g., Halite NaCl). Soluble.
    \item \textbf{Covalent}: Sharing of electrons. Strongest bond (e.g., Diamond).
    \item \textbf{Metallic}: Electrons drift freely ("sea of electrons"). Conductive, malleable (e.g., Gold, Copper).
    \item \textbf{Van der Waals}: Weak residual forces between sheets (e.g., Graphite).
\end{itemize}

\subsection{Physical Properties}
\begin{itemize}
    \item \textbf{Lustre}: Appearance in light (Metallic vs Non-metallic).
    \item \textbf{Color}: Diagnostic for some, unreliable for others (e.g., Quartz varies due to impurities).
    \item \textbf{Streak}: Color of powdered mineral (Streak plate).
    \item \textbf{Hardness}: Resistance to scratching (Mohs Scale).
        \begin{itemize}
            \item 1 (Softest): Talc. 2.5: Fingerprint. 10 (Hardest): Diamond.
        \end{itemize}
    \item \textbf{Cleavage vs Fracture}: Breaking along planes of weakness (Cleavage) vs random (Fracture/Conchoidal).
    \item \textbf{Specific Gravity}: Density ratio.
    \item \textbf{Special Properties}: 
        \begin{itemize}
            \item \textbf{Calcite}: Reacts with HCl (fizzes).
            \item \textbf{Magnetite}: Magnetic.
            \item \textbf{Halite}: Salty taste.
        \end{itemize}
    \item \textbf{Polymorphs}: Same composition, different structure/properties. (Diamond [High P] vs Graphite [Low P] - both Carbon).
\end{itemize}

\subsection{Mineral Groups}
\begin{itemize}
    \item \textbf{Silicates}: Most common group (>90\% of crust). Based on Silicon-Oxygen Tetrahedron (SiO$_4^{4-}$).
        \begin{itemize}
            \item \textbf{Mafic (Dark)}: Rich in Fe, Mg. High density, high melt temp. (Olivine, Pyroxene, Amphibole, Biotite).
            \item \textbf{Felsic (Light)}: Rich in Si, Al, K, Na. Low density, low melt temp. (Quartz, Feldspar, Muscovite).
        \end{itemize}
    \item \textbf{Non-Silicates}:
        \begin{itemize}
            \item \textbf{Carbonates}: Calcite, Dolomite (Limestone, Marble).
            \item \textbf{Halides}: Halite (Salt), Fluorite.
            \item \textbf{Oxides}: Hematite, Magnetite (Ores).
            \item \textbf{Sulfides}: Galena, Pyrite (Ores).
            \item \textbf{Native Elements}: Gold, Copper, Sulfur.
        \end{itemize}
    \item \textbf{Rock-Forming} (Abundant) vs \textbf{Economic} (Valuable).
\end{itemize}

\section{Magma \& Igneous Rocks}
\subsection{Magma Components}
\begin{itemize}
    \item \textbf{Melt}: Liquid portion.
    \item \textbf{Solids}: Crystallized minerals.
    \item \textbf{Volatiles}: Gases (H$_2$O, CO$_2$, SO$_2$) dissolved in melt. Escape at surface.
\end{itemize}

\subsection{Generation of Magma (Melting)}
\begin{itemize}
    \item \textbf{Geothermal Gradient}: Temp increases w/ depth (~25$^\circ$C/km). Rock in mantle is solid (hot but high P).
    \item \textbf{Decompression Melting}: Pressure drops as rock rises, lowering melting point. (Divergent boundaries, Hot Spots).
    \item \textbf{Flux (Volatile) Melting}: Water released from subducting slab lowers melting point of mantle wedge. (Subduction zones).
    \item \textbf{Heat Transfer}: Rising magma melts surrounding crustal rock.
\end{itemize}

\subsection{Evolution of Magma}
\begin{itemize}
    \item \textbf{Differentiation}: Separation of components.
    \item \textbf{Crystal Settling}: Early formed crystals (heavy) sink.
    \item \textbf{Assimilation}: Melting surrounding host rock.
    \item \textbf{Magma Mixing}: Two bodies join.
    \item \textbf{Bowen's Reaction Series}: Predictable order of crystallization.
        \begin{itemize}
            \item \textbf{Discontinuous}: Olivine (High T) $\rightarrow$ Pyroxene $\rightarrow$ Amphibole $\rightarrow$ Biotite.
            \item \textbf{Continuous}: Ca-rich Plagioclase $\rightarrow$ Na-rich Plagioclase.
            \item \textbf{Last to Crystallize (Low T)}: K-Feldspar, Muscovite, Quartz.
        \end{itemize}
\end{itemize}

\subsection{Igneous Classifications}
\begin{itemize}
    \item \textbf{Intrusive (Plutonic)}: Cools slowly underground. Large crystals.
    \item \textbf{Extrusive (Volcanic)}: Cools quickly on surface. Small crystals.
\end{itemize}

\subsection{Intrusive Structures (Plutons)}
\begin{itemize}
    \item \textbf{Dikes:} cut across layers (Vertically)
    \item \textbf{Sills:} cut through layers (Horizontally)
    \item \textbf{Batholith:} Deep-seated magma chambers that cool slowly underground.  (Huge)
    \item \textbf{Laccolith:} Smaller version that looks like a dome.
\end{itemize}

\subsection{Textures (Crystal Size)}
\begin{itemize}
    \item \textbf{Phaneritic}: Coarse-grained (Visible). Slow cooling (Intrusive).
    \item \textbf{Aphanitic}: Fine-grained (Microscopic). Fast cooling (Extrusive).
    \item \textbf{Porphyritic}: Large crystals (Phenocrysts) in fine matrix (Groundmass). Two-stage cooling, and the result of slow cooling followed by a sudden increase in the cooling rate..
    \item \textbf{Glassy}: No crystals (Obsidian). Instant cooling.
    \item \textbf{Vesicular}: Holes from gas bubbles (Pumice, Scoria).
    \item \textbf{Pyroclastic}: Fragmented rock/ash (Tuff).
\end{itemize}

\subsection{Compositions}
\begin{itemize}
    \item \textbf{Felsic} (Granitic): High Silica (>65\%). Light color. High Viscosity.
        \begin{itemize}
            \item Intrusive: \textbf{Granite}. Extrusive: \textbf{Rhyolite}.
        \end{itemize}
    \item \textbf{Intermediate} (Andesitic): Moderate Silica. Gray color.
        \begin{itemize}
            \item Intrusive: \textbf{Diorite}. Extrusive: \textbf{Andesite}.
        \end{itemize}
    \item \textbf{Mafic} (Basaltic): Low Silica. High Fe/Mg. Dark color. Low Viscosity.
        \begin{itemize}
            \item Intrusive: \textbf{Gabbro}. Extrusive: \textbf{Basalt}.
        \end{itemize}
    \item \textbf{Ultramafic}: Very low Silica. Green (Olivine). Mantle rock.
        \begin{itemize}
            \item Intrusive: \textbf{Peridotite}.
        \end{itemize}
\end{itemize}

\section{Volcanoes \& Hazards}
\subsection{Viscosity \& Eruptions}
\begin{itemize}
    \item \textbf{Viscosity}: Resistance to flow. Controls explosivity.
        \begin{itemize}
            \item \textbf{High Viscosity}: High Silica (Felsic), Low Temp. Gas trapped $\rightarrow$ \textbf{Explosive}.
            \item \textbf{Low Viscosity}: Low Silica (Mafic), High Temp. Gas escapes $\rightarrow$ \textbf{Effusive}.
            \item \textbf{Simplified Rule:} Subduction/Explosive $\rightarrow$ Stratovolcanoes. Hotspots $\rightarrow$ Shield/Effusive.
        \end{itemize}
\end{itemize}

\subsection{Volcano Types}
\begin{itemize}
    \item \textbf{Shield Volcano}: Broad, slight dome. Fluid Basaltic lava. Liquid flows. (e.g., Mauna Loa, Hawaii).
    \item \textbf{Cinder Cone}: Small, steep. Built from ejected lava fragments (scoria). Short-lived. (e.g., Paricutin).
    \item \textbf{Composite Cone (Stratovolcano)}: Large, classic cone. Interbedded lava/pyroclastics. Violent. Ring of Fire. (e.g., Mt. Rainier, St. Helens, Fuji).
    \item \textbf{Caldera}: Large collapse depression (>1 km). (e.g., Crater Lake, Yellowstone).
    \item \textbf{Flood Basalts}: Massive volume of fluid lava from fissures. (e.g., Columbia River Basalts).
\end{itemize}

\subsection{Volcanic Hazards}
\begin{itemize}
    \item \textbf{Pyroclastic Flow}: Superheated gas/ash avalanche. >100 km/h. Most deadly.
    \item \textbf{Lahars}: Volcanic mudflows (Ash + Water). Concrete-like. Can occur without eruption (rain/melt). Major threat to valleys.
    \item \textbf{Ejecta / Tephra}: Ash (fine), Lapilli (gravel), Blocks (hard), Bombs (molten-streamlined).
    \item \textbf{Lava Flows}: Property damage, usually slow. Pahoehoe (ropy, hotter, less viscous), A'a (jagged, otherwise same compistion).
    \item \textbf{Gas}: H$_2$O, CO$_2$, SO$_2$. CO$_2$ is denser than air, collects in lows (Lake Nyos).
\end{itemize}

\subsection{Monitoring \& Status}
\begin{itemize}
    \item \textbf{Active}: Erupting or recent. \textbf{Dormant}: Sleeping, capable. \textbf{Extinct}: No magma source.
    \item \textbf{Signals}: Earthquakes (magma movement), Ground deformation (inflation), Gas emission changes.
\end{itemize}

\section{Ocean Floor}
\begin{itemize}
    \item \textbf{Bathymetry}: Mapping ocean depth (Sonar).
    \item \textbf{Continental Margin}: Transition from land to deep sea.
        \begin{itemize}
            \item \textbf{Shelf}: Flooded continent, gentle slope.
            \item \textbf{Slope}: Steep drop-off.
            \item \textbf{Rise}: Sediment accumulation.
        \end{itemize}
    \item \textbf{Deep Ocean Basin}:
        \begin{itemize}
            \item \textbf{Abyssal Plain}: Flat, sediment covered.
            \item \textbf{Trenches}: Deepest points (subduction).
            \item \textbf{Seamounts}: Submerged volcanoes.
        \end{itemize}
    \item \textbf{Mid-Ocean Ridge}: Longest topo feature. Rift valley in center.
\end{itemize}

\end{multicols*}
\end{document}
