\documentclass[8pt,landscape]{extarticle}
\usepackage[utf8]{inputenc}
\usepackage[margin=0.3in]{geometry}
\usepackage{multicol}
\usepackage{enumitem}
\usepackage{titlesec}
\usepackage{xcolor}
\usepackage{nopageno}

% Compact list settings
\setlist{nosep, leftmargin=*}
\setlist[itemize]{label=\textbullet}

% Compact section titles
\titleformat{\section}
  {\normalfont\bfseries\uppercase}{\thesection}{1em}{}
\titlespacing*{\section}{0pt}{1ex plus 1ex minus .2ex}{0.5ex plus .2ex}

\titleformat{\subsection}
  {\normalfont\bfseries}{\thesubsection}{1em}{}
\titlespacing*{\subsection}{0pt}{0.5ex plus 1ex minus .2ex}{0.2ex plus .2ex}

\setlength{\parindent}{0pt}
\setlength{\parskip}{0pt}

\begin{document}

\begin{multicols*}{4}

\begin{center}
    \textbf{\large GEOLOGY EXAM 1 CHEAT SHEET}
\end{center}

\section{Introduction to Geology}
\subsection{Basics}
\begin{itemize}
    \item \textbf{Geology}: Science of Earth processes, composition, structure, history.
    \item \textbf{Physical Geology}: Materials \& processes.
    \item \textbf{Historical Geology}: Origin \& development through time.
    \item \textbf{Uniformitarianism}: "The present is the key to the past." Processes today (erosion, volcanism) operated in the past.
    \item \textbf{Scientific Method}: Observation $\rightarrow$ Hypothesis (testable) $\rightarrow$ Theory (well-tested). Data supports/disproves hypotheses.
\end{itemize}

\subsection{Origin of Earth}
\begin{itemize}
    \item \textbf{Big Bang}: 13.7 billion years ago (Universe formation).
    \item \textbf{Nebular Theory}: Solar system formed from rotating cloud (solar nebula).
    \item \textbf{Solar System Age}: 4.6 billion years.
    \item \textbf{Earth Formation}: 4.6 billion years ago. Accretion of planetesimals.
    \item \textbf{Differentiation}: Melting caused dense Fe/Ni to sink (Core), lighter mix to float (Crust/Mantle).
\end{itemize}

\subsection{Geologic Time}
\begin{itemize}
    \item Scale: Eons $\rightarrow$ Eras $\rightarrow$ Periods $\rightarrow$ Epochs.
    \item \textbf{Human Appearance}: Very recent (last few mins of "Earth Year").
\end{itemize}

\section{Earth Systems \& Structure}
\subsection{Spheres}
\begin{itemize}
    \item \textbf{Geosphere}: Solid Earth (Crust, Mantle, Core).
    \item \textbf{Hydrosphere}: Water (Oceans, rivers).
    \item \textbf{Atmosphere}: Gases. \textbf{Biosphere}: Life.
\end{itemize}

\subsection{Internal Structure (Composition)}
\begin{itemize}
    \item \textbf{Crust}: Rocky outer skin.
        \begin{itemize}
            \item \textbf{Oceanic}: Thin ($\sim$7km), Denser (3.0 g/cm$^3$), Basaltic (Mafic).
            \item \textbf{Continental}: Thick (35-70km), Less Dense (2.7 g/cm$^3$), Granitic (Felsic).
        \end{itemize}
    \item \textbf{Mantle}: Solid rocky shell (Peridotite). More dense than crust.
    \item \textbf{Core}: Iron-Nickel alloy.
        \begin{itemize}
            \item \textbf{Outer Core}: Liquid (generates Magnetic Field). S-waves cannot pass.
            \item \textbf{Inner Core}: Solid (due to immense pressure).
        \end{itemize}
\end{itemize}

\subsection{Properties vs Composition}
\begin{itemize}
    \item \textbf{Lithosphere}: Rigid outer layer (Crust + Upper Mantle). Forms tectonic plates.
    \item \textbf{Asthenosphere}: Weak, ductile layer below lithosphere (Upper Mantle). Allows plates to move.
\end{itemize}

\section{Plate Tectonics}
\subsection{Key Concepts}
\begin{itemize}
    \item \textbf{Hypothesis}: Continental Drift (Wegener) $\rightarrow$ Plate Tectonics (1960s).
    \item \textbf{Driving Force}: Convection in mantle (ridge push, slab pull).
    \item \textbf{Rate}: $\sim$3 cm/year (fingerprint growth speed).
\end{itemize}

\subsection{Boundaries}
\begin{itemize}
    \item \textbf{Divergent} (Move apart):
        \begin{itemize}
            \item Mid-Ocean Ridges (Seafloor spreading). New crust created.
            \item Rift Valleys (e.g., East Africa, Mt. Kilimanjaro).
            \item Decompression melting.
            \item Example: Iceland, Mid-Atlantic Ridge.
        \end{itemize}
    \item \textbf{Convergent} (Move together):
        \begin{itemize}
            \item \textbf{Ocean-Cont}: Subduction. Volcanic Arc (e.g., Cascades, Andes). Trench.
            \item \textbf{Ocean-Ocean}: Subduction. Island Arc (e.g., Japan, Aleutians). Trench.
            \item \textbf{Cont-Cont}: Collision. Mountain Belt (e.g., Himalayas). No subduction/volcanism.
            \item Oceanic lithosphere sinks at trenches.
        \end{itemize}
    \item \textbf{Transform} (Slide past):
        \begin{itemize}
            \item No crust created/destroyed.
            \item Example: San Andreas Fault.
        \end{itemize}
\end{itemize}

\subsection{Evidence}
\begin{itemize}
    \item \textbf{Hot Spots}: Stationary mantle plumes. Plate moves over them creating track (e.g., Hawaii). Shows direction of plate motion.
    \item \textbf{Paleomagnetism}: Earth's field reverses. Recorded in oceanic crust. Symmetric stripes at ridges prove seafloor spreading.
    \item \textbf{Declination}: Angle to pole. \textbf{Inclination}: Angle to horizontal (latitude).
\end{itemize}

\section{Minerals}
\subsection{Definition}
\begin{itemize}
    \item Naturally occurring, Inorganic, Solid, Ordered internal structure, Definite chemical composition.
    \item \textbf{Rock}: Aggregate of minerals.
\end{itemize}

\subsection{Formation}
\begin{itemize}
    \item Crystallization from magma.
    \item Precipitation from water.
    \item Weathering (secondary minerals).
    \item Metamorphism (heat/pressure).
    \item Open space allows good crystal faces.
\end{itemize}

\subsection{Properties}
\begin{itemize}
    \item \textbf{Color}: Variable (impurities). Unreliable for Quartz.
    \item \textbf{Shape}: Determined by crystal structure (e.g., Quartz pyramid).
    \item \textbf{Polymorphs}: Same composition, different structure (Diamond vs Graphite). Diamond=High P, Graphite=Low P.
\end{itemize}

\subsection{Classes}
\begin{itemize}
    \item \textbf{Silicates}: Most common (SiO$_4$ tetrahedron).
        \begin{itemize}
            \item \textbf{Felsic (Light)}: Quartz, Feldspar, Muscovite. Low density, low melt temp. Continental.
            \item \textbf{Mafic (Dark)}: Olivine, Pyroxene, Biotite, Amphibole. High Fe/Mg, high density, high melt temp. Oceanic/Mantle.
        \end{itemize}
    \item \textbf{Non-Silicates}: Carbonates, Halides, Oxides, Native elements.
    \item \textbf{Economic}: Less abundant, specific uses. \textbf{Rock-forming}: Abundant.
\end{itemize}

\section{Magma \& Igneous Rocks}
\subsection{Magma Formation (Melting)}
\begin{itemize}
    \item \textbf{Decompression}: P drops as rock rises (Divergent, Hot Spots).
    \item \textbf{Volatile (Flux)}: Water lowers melting point (Subduction).
    \item \textbf{Heat Transfer}: Rising magma melts crust (Continental rifts/Hot spots).
\end{itemize}

\subsection{Properties}
\begin{itemize}
    \item \textbf{Viscosity}: Resistance to flow.
        \begin{itemize}
            \item High Temp $\rightarrow$ Low Viscosity.
            \item High Silica (Felsic) $\rightarrow$ High Viscosity.
            \item Low Silica (Mafic) $\rightarrow$ Low Viscosity.
        \end{itemize}
    \item \textbf{Rising}: Magma rises because it is less dense than surrounding rock and under pressure.
\end{itemize}

\subsection{Textures (Cooling Rate)}
\begin{itemize}
    \item \textbf{Aphanitic} (Fine): Fast cooling (Extrusive/Volcanic).
    \item \textbf{Phaneritic} (Coarse): Slow cooling (Intrusive/Plutonic).
    \item \textbf{Porphyritic}: Two stages (Slow then Fast). Phenocrysts (large) in matrix.
    \item \textbf{Glassy}: Very fast cooling (Obsidian).
    \item \textbf{Pyroclastic}: Explosive fragments (Tuffs).
    \item \textbf{Vesicular}: Gas bubbles (Pumice/Scoria).
\end{itemize}

\subsection{Composition}
\begin{itemize}
    \item \textbf{Felsic}: High Si, Na, K. Granite (Intr) / Rhyolite (Extr).
    \item \textbf{Intermediate}: Andesite (Extr) / Diorite (Intr).
    \item \textbf{Mafic}: High Fe, Mg, Ca. Basalt (Extr) / Gabbro (Intr).
    \item \textbf{Ultramafic}: Peridotite (Mantle).
\end{itemize}

\section{Volcanoes}
\subsection{Types}
\begin{itemize}
    \item \textbf{Shield}: Broad, gentle slopes. Fluid basalt. Effusive. (Mauna Loa).
    \item \textbf{Stratovolcano (Composite)}: Large, cone-shaped. Andesite/Rhyolite. Explosive. (Mt. Rainer, Fuji).
    \item \textbf{Cinder Cone}: Small, steep. Pyroclastic ejecta.
    \item \textbf{Caldera}: Collapsed summit (Crater Lake).
\end{itemize}

\subsection{Hazards}
\begin{itemize}
    \item \textbf{Lahars}: Mudflows (Ash + Water). Flow far, follow rivers. Can occur without eruption.
    \item \textbf{Pyroclastic Flows}: Hot gas/ash avalanche. Very fast (>100 km/h), deadly.
    \item \textbf{Lava Flows}: Destroys property, usually slow enough to escape (except fluid basalt).
    \item \textbf{Ash}: Buries landscape, engine failure, respiratory issues.
    \item \textbf{Gas}: SO$_2$, CO$_2$. Can suffocate.
\end{itemize}

\subsection{Eruption Styles}
\begin{itemize}
    \item \textbf{Effusive}: Low viscosity (Basalt). Pahoehoe (ropy), A'a (jagged).
    \item \textbf{Explosive}: High viscosity (Felsic/Intermediate). Trapped gas builds pressure.
    \item \textbf{Flood Basalts}: Massive flows, low viscosity, fissures (Columbia River).
\end{itemize}

\subsection{Tectonic Settings}
\begin{itemize}
    \item \textbf{Ring of Fire}: Subduction zones (Convergent). Most volcanoes here.
    \item \textbf{Mid-Ocean Ridges}: Divergent. Most lava production (underwater).
    \item \textbf{Hot Spots}: Intra-plate (Hawaii, Yellowstone).
\end{itemize}

\section{Rock Cycle}
\begin{itemize}
    \item \textbf{Igneous}: Melt $\rightarrow$ Cool/Crystallize.
    \item \textbf{Sedimentary}: Weathering $\rightarrow$ Sediment $\rightarrow$ Lithification.
    \item \textbf{Metamorphic}: Heat/Pressure $\rightarrow$ Solid state change.
    \item Any rock can become any other rock given the right process.
\end{itemize}

\section{Ocean Floor Features}
\begin{itemize}
    \item \textbf{Continental Margin}: Shelf (flooded extension), Slope (drop-off), Rise (sediment wedge).
    \item \textbf{Abyssal Plain}: Flat deep ocean floor.
    \item \textbf{Trench}: Deepest parts, at subduction zones.
    \item \textbf{Seamounts}: Underwater volcanoes (Hot spots/Ridges).
\end{itemize}

\end{multicols*}
\end{document}
