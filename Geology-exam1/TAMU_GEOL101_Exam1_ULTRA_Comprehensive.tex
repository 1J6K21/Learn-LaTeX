
\documentclass[10pt]{article}
\usepackage[margin=0.5in]{geometry}
\usepackage{multicol}
\usepackage{array}
\usepackage{longtable}
\usepackage{booktabs}
\usepackage{enumitem}
\setlist{nosep}
\pagestyle{empty}

\begin{document}
\begin{center}
{\Large TAMU GEOL 101 – Comprehensive Exam I Master Sheet (Full Content Mode)}
\end{center}

\begin{multicols}{2}

%%%%%%%%%%%%%%%%%%%%%%%%%%%%%%%%%%%%%%%%%%%%%%%%%%%%%%%%%%%%
\section*{EARTH STRUCTURE}

\textbf{Compositional Layers}
\begin{itemize}
\item Crust – continental (felsic, granitic, low density), oceanic (mafic, basaltic, higher density)
\item Mantle – ultramafic (peridotite), Fe-Mg rich
\item Core – Fe-Ni metallic
\end{itemize}

\textbf{Mechanical Layers}
\begin{itemize}
\item Lithosphere – rigid (crust + uppermost mantle)
\item Asthenosphere – plastic/ductile, partial melt
\item Mesosphere – lower mantle
\item Outer core – liquid
\item Inner core – solid
\end{itemize}

\textbf{Density Trends}
\begin{itemize}
\item Density increases with depth
\item Continental crust $\sim$2.7 g/cm$^3$
\item Oceanic crust $\sim$3.0 g/cm$^3$
\item Mantle $\sim$3.3+ g/cm$^3$
\item Core $>$10 g/cm$^3$
\end{itemize}

%%%%%%%%%%%%%%%%%%%%%%%%%%%%%%%%%%%%%%%%%%%%%%%%%%%%%%%%%%%%
\section*{MINERALS}

\textbf{Definition}
Naturally occurring, inorganic, solid, definite chemical composition, ordered internal structure.

\textbf{Silicate Structure}
\begin{itemize}
\item Silicon-Oxygen Tetrahedron (SiO$_4^{4-}$)
\item Polymerization increases silica content and viscosity
\end{itemize}

\textbf{Common Silicates}
\begin{itemize}
\item Olivine – mafic
\item Pyroxene – mafic
\item Amphibole – intermediate
\item Biotite – sheet silicate
\item Muscovite – felsic
\item Feldspars – most abundant
\item Quartz – high silica
\end{itemize}

\textbf{Physical Properties}
\begin{itemize}
\item Hardness (Mohs scale)
\item Cleavage vs Fracture
\item Luster
\item Streak
\item Density
\item Crystal form
\end{itemize}

%%%%%%%%%%%%%%%%%%%%%%%%%%%%%%%%%%%%%%%%%%%%%%%%%%%%%%%%%%%%
\section*{IGNEOUS PROCESSES}

\textbf{Magma Formation Mechanisms}
\begin{itemize}
\item Decompression melting (divergent boundaries, hotspots)
\item Flux melting (subduction zones – volatiles lower melting point)
\item Heat transfer melting
\end{itemize}

\textbf{Bowen's Reaction Series}
\begin{itemize}
\item Discontinuous: Olivine $\rightarrow$ Pyroxene $\rightarrow$ Amphibole $\rightarrow$ Biotite
\item Continuous: Ca-rich Plagioclase $\rightarrow$ Na-rich Plagioclase
\item Final: K-Feldspar, Muscovite, Quartz
\end{itemize}

%%%%%%%%%%%%%%%%%%%%%%%%%%%%%%%%%%%%%%%%%%%%%%%%%%%%%%%%%%%%
\section*{BASALTIC vs FELSIC MAGMA}

\begin{longtable}{p{2.5cm}p{3cm}p{3cm}}
\toprule
Property & Basaltic (Mafic) & Felsic \\
\midrule
Silica & Low (45–52\%) & High (65–75\%) \\
Temperature & High (1000–1200°C) & Lower (650–800°C) \\
Viscosity & Low & High \\
Gas Content & Low & High \\
Eruption Style & Effusive & Explosive \\
Rock Type & Basalt & Rhyolite/Granite \\
Color & Dark & Light \\
Density & Higher & Lower \\
\bottomrule
\end{longtable}

%%%%%%%%%%%%%%%%%%%%%%%%%%%%%%%%%%%%%%%%%%%%%%%%%%%%%%%%%%%%
\section*{VOLCANO TYPES}

\textbf{Shield Volcano}
\begin{itemize}
\item Broad, gentle slopes
\item Basaltic lava
\item Effusive eruptions
\end{itemize}

\textbf{Composite (Stratovolcano)}
\begin{itemize}
\item Alternating lava + pyroclastics
\item Felsic/intermediate
\item Explosive
\end{itemize}

\textbf{Cinder Cone}
\begin{itemize}
\item Pyroclastic fragments
\item Steep slopes
\item Short-lived
\end{itemize}

%%%%%%%%%%%%%%%%%%%%%%%%%%%%%%%%%%%%%%%%%%%%%%%%%%%%%%%%%%%%
\section*{PYROCLASTIC MATERIAL}

\begin{itemize}
\item Ash ($<$2 mm)
\item Lapilli (2–64 mm)
\item Bombs/Blocks ($>$64 mm)
\item Pyroclastic flow – hot, fast-moving gas + fragments
\item Lahars – volcanic mudflows
\end{itemize}

%%%%%%%%%%%%%%%%%%%%%%%%%%%%%%%%%%%%%%%%%%%%%%%%%%%%%%%%%%%%
\section*{PLATE TECTONICS}

\textbf{Boundary Types}
\begin{itemize}
\item Divergent – mid-ocean ridges, rifting
\item Convergent – subduction or collision
\item Transform – strike-slip motion
\end{itemize}

\textbf{Ocean-Continent Convergence}
\begin{itemize}
\item Oceanic plate subducts
\item Volcanic arc forms
\item Deep ocean trench
\end{itemize}

\textbf{Ocean-Ocean Convergence}
\begin{itemize}
\item Island arc
\end{itemize}

\textbf{Continent-Continent}
\begin{itemize}
\item Mountain building
\item No subduction (buoyant crust)
\end{itemize}

%%%%%%%%%%%%%%%%%%%%%%%%%%%%%%%%%%%%%%%%%%%%%%%%%%%%%%%%%%%%
\section*{CHEMICAL vs PHYSICAL BEHAVIOR}

\textbf{Physical Weathering}
\begin{itemize}
\item Frost wedging
\item Exfoliation
\item Abrasion
\end{itemize}

\textbf{Chemical Weathering}
\begin{itemize}
\item Hydrolysis
\item Oxidation
\item Dissolution
\item Increases in warm, wet climates
\end{itemize}

%%%%%%%%%%%%%%%%%%%%%%%%%%%%%%%%%%%%%%%%%%%%%%%%%%%%%%%%%%%%
\section*{ISCHEMATIC CONCEPTS}

\textbf{Isostasy}
\begin{itemize}
\item Buoyant equilibrium of crust
\item Thicker crust = deeper root
\end{itemize}

\textbf{Seafloor Spreading}
\begin{itemize}
\item Symmetrical magnetic stripes
\item Youngest rock at ridge
\item Age increases away from ridge
\end{itemize}

%%%%%%%%%%%%%%%%%%%%%%%%%%%%%%%%%%%%%%%%%%%%%%%%%%%%%%%%%%%%
\section*{KEY EXAM DISTINCTIONS}

\begin{itemize}
\item Mafic = Mg + Fe rich
\item Felsic = Feldspar + Silica rich
\item Viscosity controlled by silica + temperature + volatiles
\item Density differences drive subduction
\item Explosivity tied to trapped gas + viscosity
\end{itemize}

\end{multicols}
\end{document}
