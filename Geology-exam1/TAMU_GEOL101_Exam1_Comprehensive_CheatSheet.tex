
\documentclass[9pt]{article}
\usepackage[margin=0.5in]{geometry}
\usepackage{multicol}
\usepackage{array}
\usepackage{booktabs}
\usepackage{enumitem}
\usepackage{titlesec}
\usepackage{amsmath}
\setlength{\parindent}{0pt}
\setlength{\parskip}{2pt}
\renewcommand{\arraystretch}{1.15}
\pagestyle{empty}
\titlespacing*{\section}{0pt}{4pt}{4pt}

\begin{document}
\begin{center}
{\LARGE \textbf{TAMU GEOL 101 – Comprehensive Exam 1 Cheat Sheet}}
\end{center}

\begin{multicols}{3}

%%%%%%%%%%%%%%%%%%%%%%%%%%%%%%%%%%%%%%%%%%%%%%%%%%%%%
\section*{ORIGIN OF UNIVERSE \& EARTH}

\textbf{Big Bang:} 13.7 Ga (Universe forms) \\
\textbf{Nebular Theory:} Solar system from rotating solar nebula \\
\textbf{Earth age:} 4.6 Ga \\

\textbf{Formation sequence:}
Collapse → Spinning disk → Proto-sun → Planetesimals → Planets

\textbf{Chemical differentiation:}
Dense Fe-Ni sank → Core formed → Mantle + crust layered by density

%%%%%%%%%%%%%%%%%%%%%%%%%%%%%%%%%%%%%%%%%%%%%%%%%%%%%
\section*{EARTH SYSTEMS}

\begin{tabular}{@{}p{2.3cm}p{2.5cm}@{}}
\toprule
Geosphere & Solid Earth \\
Hydrosphere & Liquid water \\
Atmosphere & Gases \\
Cryosphere & Ice \\
Biosphere & Life \\
\bottomrule
\end{tabular}

\textbf{Steady State:} Input = Output \\
Output > Input → depletion \\
Input > Output → accumulation \\

\textbf{Uniformitarianism:} Present = key to past

%%%%%%%%%%%%%%%%%%%%%%%%%%%%%%%%%%%%%%%%%%%%%%%%%%%%%
\section*{EARTH STRUCTURE (COMPOSITIONAL)}

\begin{tabular}{@{}p{2cm}p{3cm}@{}}
\toprule
Crust & Oceanic: Basalt, 8 km, 3.0 g/cm$^3$ \\
 & Continental: Granite, 30–70 km, 2.7 g/cm$^3$ \\
Mantle & Peridotite (largest by volume) \\
Outer Core & Liquid Fe-Ni \\
Inner Core & Solid (pressure) \\
\bottomrule
\end{tabular}

\textbf{Thickest sphere:} Geosphere \\
\textbf{Geothermal gradient:} Temp ↑ with depth

%%%%%%%%%%%%%%%%%%%%%%%%%%%%%%%%%%%%%%%%%%%%%%%%%%%%%
\section*{PHYSICAL LAYERS}

Lithosphere = crust + upper mantle (rigid plates) \\
Asthenosphere = ductile/plastic \\
Mesosphere = lower mantle \\
Boundary (lith/asth) = temperature-controlled

%%%%%%%%%%%%%%%%%%%%%%%%%%%%%%%%%%%%%%%%%%%%%%%%%%%%%
\section*{SEISMIC WAVES}

\begin{tabular}{@{}p{2cm}p{3cm}@{}}
\toprule
P-wave & Solids + liquids \\
S-wave & Solids only \\
\bottomrule
\end{tabular}

No S-wave in outer core → liquid

%%%%%%%%%%%%%%%%%%%%%%%%%%%%%%%%%%%%%%%%%%%%%%%%%%%%%
\section*{PLATE TECTONICS}

\begin{tabular}{@{}p{2.2cm}p{2.8cm}@{}}
\toprule
Divergent & Ridge, rift valley, decompression melting \\
Convergent (O-C) & Subduction, volcanoes on continent \\
Convergent (C-C) & Mountain belts, no volcanism \\
Transform & Shear, no creation/destruction \\
\bottomrule
\end{tabular}

Drivers: Slab pull, Ridge push, Convection \\
Rate: ~3 cm/yr \\
Old seafloor < 200 Ma (recycled)

Stress types: Tension (rift), Compression (convergent), Shear (transform)

%%%%%%%%%%%%%%%%%%%%%%%%%%%%%%%%%%%%%%%%%%%%%%%%%%%%%
\section*{VOLCANISM}

\begin{tabular}{@{}p{2cm}p{3cm}@{}}
\toprule
Shield & Mafic, low viscosity, effusive \\
Stratovolcano & Felsic/intermediate, explosive \\
Caldera & Chamber collapse \\
Hot spot & Mantle plume track \\
\bottomrule
\end{tabular}

Most volcanoes: Convergent-subduction \\

Melting types:
Decompression → ridges/hot spots \\
Volatile melting → subduction \\
Heat-transfer → continental crust

Lahars = water + ash debris flows \\
CO$_2$ hazard = colorless, dense, suffocation

%%%%%%%%%%%%%%%%%%%%%%%%%%%%%%%%%%%%%%%%%%%%%%%%%%%%%
\section*{MAGMA PROPERTIES}

\begin{tabular}{@{}p{2cm}p{3cm}@{}}
\toprule
Temp ↑ & Viscosity ↓ \\
Silica ↑ & Viscosity ↑ \\
Volatiles ↑ & Explosivity ↑ \\
\bottomrule
\end{tabular}

Silica content:
Ultramafic 38–45\% \\
Mafic 45–52\% \\
Intermediate 52–66\% \\
Felsic 66–76\% \\

Magma temp: 650–1100°C

%%%%%%%%%%%%%%%%%%%%%%%%%%%%%%%%%%%%%%%%%%%%%%%%%%%%%
\section*{BASALTIC vs FELSIC}

\begin{tabular}{@{}p{1.8cm}p{1.8cm}p{1.8cm}@{}}
\toprule
 & Basaltic & Felsic \\
\midrule
Silica & Low & High \\
Viscosity & Low & High \\
Density & High & Low \\
Temp & High & Lower \\
Volcano & Shield & Stratovolcano \\
Color & Dark & Light \\
\bottomrule
\end{tabular}

Mafic = Mg, Fe rich \\
Felsic = Si, Al rich

%%%%%%%%%%%%%%%%%%%%%%%%%%%%%%%%%%%%%%%%%%%%%%%%%%%%%
\section*{ROCK TYPES}

Igneous = cooling magma/lava \\
Sedimentary = lithified sediment \\
Metamorphic = heat + pressure (solid state)

Metamorphic grade:
Slate → Phyllite → Schist → Gneiss

Sandstone → Quartzite \\
Limestone → Marble

%%%%%%%%%%%%%%%%%%%%%%%%%%%%%%%%%%%%%%%%%%%%%%%%%%%%%
\section*{WEATHERING \& SEDIMENT}

Mechanical = frost wedging \\
Chemical = acid reaction \\
Rainwater + CO$_2$ → carbonic acid

Grain size:
Gravel > Sand > Silt > Clay

Glacial deposits = very poorly sorted

%%%%%%%%%%%%%%%%%%%%%%%%%%%%%%%%%%%%%%%%%%%%%%%%%%%%%
\section*{FAULTS \& DEFORMATION}

Fault = fracture + displacement \\
Joint = no displacement \\
Elastic deformation = returns to shape \\
Brittle deformation = faults

Reverse fault = compression \\
Normal fault = tension

Strain = result of stress

%%%%%%%%%%%%%%%%%%%%%%%%%%%%%%%%%%%%%%%%%%%%%%%%%%%%%
\section*{KEY NUMBERS}

Earth age: 4.6 Ga \\
Universe age: 13.7 Ga \\
Oceanic crust density: ~3.0 g/cm$^3$ \\
Continental crust density: ~2.7 g/cm$^3$ \\

\end{multicols}
\end{document}
